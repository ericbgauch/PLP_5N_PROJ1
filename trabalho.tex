%% abtex2-modelo-trabalho-academico.tex, v-1.9.2 laurocesar
%% Copyright 2012-2014 by abnTeX2 group at http://abntex2.googlecode.com/ 
%%
%% This work may be distributed and/or modified under the
%% conditions of the LaTeX Project Public License, either version 1.3
%% of this license or (at your option) any later version.
%% The latest version of this license is in
%%   http://www.latex-project.org/lppl.txt
%% and version 1.3 or later is part of all distributions of LaTeX
%% version 2005/12/01 or later.
%%
%% This work has the LPPL maintenance status `maintained'.
%% 
%% The Current Maintainer of this work is the abnTeX2 team, led
%% by Lauro César Araujo. Further information are available on 
%% http://abntex2.googlecode.com/
%%
%% This work consists of the files abntex2-modelo-trabalho-academico.tex,
%% abntex2-modelo-include-comandos and abntex2-modelo-references.bib
%%

% ------------------------------------------------------------------------
% ------------------------------------------------------------------------
% abnTeX2: Modelo de Trabalho Academico (tese de doutorado, dissertacao de
% mestrado e trabalhos monograficos em geral) em conformidade com 
% ABNT NBR 14724:2011: Informacao e documentacao - Trabalhos academicos -
% Apresentacao
% ------------------------------------------------------------------------
% ------------------------------------------------------------------------

\documentclass[
    % -- opções da classe memoir --
    12pt,               % tamanho da fonte
    openright,          % capítulos começam em pág ímpar (insere página vazia caso preciso)
    twoside,            % para impressão em verso e anverso. Oposto a oneside
    a4paper,            % tamanho do papel. 
    % -- opções da classe abntex2 --
    %chapter=TITLE,     % títulos de capítulos convertidos em letras maiúsculas
    %section=TITLE,     % títulos de seções convertidos em letras maiúsculas
    %subsection=TITLE,  % títulos de subseções convertidos em letras maiúsculas
    %subsubsection=TITLE,% títulos de subsubseções convertidos em letras maiúsculas
    % -- opções do pacote babel --
    brazil              % o último idioma é o principal do documento
    ]{abntex2}

% ---
% Pacotes básicos 
% ---
\usepackage{palatino}            % Usa a fonte Latin Modern          
\usepackage[T1]{fontenc}        % Selecao de codigos de fonte.
\usepackage[utf8]{inputenc}     % Codificacao do documento (conversão automática dos acentos)
\usepackage{lastpage}           % Usado pela Ficha catalográfica
\usepackage{indentfirst}        % Indenta o primeiro parágrafo de cada seção.
\usepackage{color}              % Controle das cores
\usepackage{graphicx}           % Inclusão de gráficos
\usepackage{microtype}          % para melhorias de justificação
% ---
        
% ---
% Pacotes adicionais, usados apenas no âmbito do Modelo Canônico do abnteX2
% ---
\usepackage{lipsum}             % para geração de dummy text
% ---

% ---
% Pacotes de citações
% ---
\usepackage[brazilian,hyperpageref]{backref}     % Paginas com as citações na bibl
\usepackage[alf]{abntex2cite}   % Citações padrão ABNT

% --- 
% CONFIGURAÇÕES DE PACOTES
% --- 

% ---
% Configurações do pacote backref
% Usado sem a opção hyperpageref de backref
\renewcommand{\backrefpagesname}{Citado na(s) página(s):~}
% Texto padrão antes do número das páginas
\renewcommand{\backref}{}
% Define os textos da citação
\renewcommand*{\backrefalt}[4]{
    \ifcase #1 %
        Nenhuma citação no texto.%
    \or
        Citado na página #2.%
    \else
        Citado #1 vezes nas páginas #2.%
    \fi}%
% ---

% ---
% Informações de dados para CAPA e FOLHA DE ROSTO
% ---
\titulo{Trabalho de Paradigmas de \\ Linguagens de Programação}
\autor{
    Eric Bueno Gauch
    \and
    Guilherme Gatto Gonçalves da Silva
    \and
    Bruno Estima Correia Milanesi Castanheira de Souza}
\local{Brasil}
\data{2017, v-0.0.1}
\instituicao{%
  Universidade Presbiteriana Mackenzie -- UPM
  \par
  Faculdade de Computação e Informática
  \par
  Programa de Graduação em Ciência da Computação}
\tipotrabalho{Trabalho Acadêmico}
% O preambulo deve conter o tipo do trabalho, o objetivo, 
% o nome da instituição e a área de concentração 
\preambulo{Trabalho da disciplina de Paradigma de Linguagem de Programação sobre a 
evolução das principais linguagens de programação}
% ---


% ---
% Configurações de aparência do PDF final

% alterando o aspecto da cor azul
\definecolor{blue}{RGB}{41,5,195}

% informações do PDF
\makeatletter
\hypersetup{
        %pagebackref=true,
        pdftitle={\@title}, 
        pdfauthor={\@author},
        pdfsubject={\imprimirpreambulo},
        pdfcreator={LaTeX with abnTeX2},
        pdfkeywords={abnt}{latex}{abntex}{abntex2}{trabalho acadêmico}, 
        colorlinks=true,            % false: boxed links; true: colored links
        linkcolor=black,             % color of internal links
        citecolor=blue,             % color of links to bibliography
        filecolor=magenta,              % color of file links
        urlcolor=blue,
        bookmarksdepth=4
}
\makeatother
% --- 

% --- 
% Espaçamentos entre linhas e parágrafos 
% --- 

% O tamanho do parágrafo é dado por:
\setlength{\parindent}{1.3cm}

% Controle do espaçamento entre um parágrafo e outro:
\setlength{\parskip}{0.2cm}  % tente também \onelineskip

% ---
% compila o indice
% ---
\makeindex
% ---

% ----
% Início do documento
% ----
\begin{document}

% Retira espaço extra obsoleto entre as frases.
\frenchspacing 

% ----------------------------------------------------------
% ELEMENTOS PRÉ-TEXTUAIS
% ----------------------------------------------------------
% \pretextual

% ---
% Capa
% ---
\imprimircapa
% ---

% ---
% Folha de rosto
% (o * indica que haverá a ficha bibliográfica)
% ---
\imprimirfolhaderosto*
% ---

% ---
% inserir o sumario
% ---
\clearpage
\pdfbookmark[0]{\contentsname}{toc}
\tableofcontents*
\cleardoublepage
% ---



% ----------------------------------------------------------
% ELEMENTOS TEXTUAIS
% ----------------------------------------------------------
\textual

% ----------------------------------------------------------
% Introdução (exemplo de capítulo sem numeração, mas presente no Sumário)
% ----------------------------------------------------------
\chapter*[Introdução]{Introdução}
\addcontentsline{toc}{chapter}{Introdução}
% ----------------------------------------------------------

Este primeiro trabalho da disciplina de Paradigmas de Linguagens de Programação
tem como objetivo introduzir as principais linguagens de programação em uso 
atualmente bem como  

Este documento e seu código-fonte são exemplos de referência de uso da classe
\textsf{abntex2} e do pacote \textsf{abntex2cite}. O documento 
exemplifica a elaboração de trabalho acadêmico (tese, dissertação e outros do
gênero) produzido conforme a ABNT NBR 14724:2011 \emph{Informação e documentação
- Trabalhos acadêmicos - Apresentação}.

A expressão ``Modelo Canônico'' é utilizada para indicar que \abnTeX\ não é
modelo específico de nenhuma universidade ou instituição, mas que implementa tão
somente os requisitos das normas da ABNT. Uma lista completa das normas
observadas pelo \abnTeX\ é apresentada em \citeonline{abntex2classe}.

Sinta-se convidado a participar do projeto \abnTeX! Acesse o site do projeto em
\url{http://abntex2.googlecode.com/}. Também fique livre para conhecer,
estudar, alterar e redistribuir o trabalho do \abnTeX, desde que os arquivos
modificados tenham seus nomes alterados e que os créditos sejam dados aos
autores originais, nos termos da ``The \LaTeX\ Project Public
License''\footnote{\url{http://www.latex-project.org/lppl.txt}}.

Encorajamos que sejam realizadas customizações específicas deste exemplo para
universidades e outras instituições --- como capas, folha de aprovação, etc.
Porém, recomendamos que ao invés de se alterar diretamente os arquivos do
\abnTeX, distribua-se arquivos com as respectivas customizações.
Isso permite que futuras versões do \abnTeX~não se tornem automaticamente
incompatíveis com as customizações promovidas. Consulte
\citeonline{abntex2-wiki-como-customizar} par mais informações.

Este documento deve ser utilizado como complemento dos manuais do \abnTeX\ 
\cite{abntex2classe,abntex2cite,abntex2cite-alf} e da classe \textsf{memoir}
\cite{memoir}. 

Esperamos, sinceramente, que o \abnTeX\ aprimore a qualidade do trabalho que
você produzirá, de modo que o principal esforço seja concentrado no principal:
na contribuição científica.

Equipe \abnTeX 

Lauro César Araujo


% ----------------------------------------------------------
% PARTE
% ----------------------------------------------------------
\part{Evolução das principais linguagens de programação}
% ----------------------------------------------------------

O desenvolvimento inicial de C ocorreu no AT&T Bell Labs entre 1969 e 1973.5 de 
acordo com Ritchie, o período mais criativo ocorreu em 1972. A linguagem foi chamada 
“C”, porque suas características foram obtidas a partir de uma linguagem anteriormente 
chamado de ” B”, que de acordo com a Ken Thompson era versão reduzida da linguagem de 
programação BCPL.6

A versão original PDP-11 do sistema Unix foi desenvolvido em assembly. Em 1973, com a 
adição dos tipos struct, a linguagem C tornou-se poderosa o suficiente para que a 
maior parte do kernel do Unix fosse reescrito em C. Este foi um dos primeiros núcleos 
de sistemas operacionais implementadas em uma linguagem diferente da linguagem Assembly. 
Em 1977, foram feitas novas mudanças por Ritchie e Stephen C. Johnson para facilitar a 
portabilidade do sistema operacional Unix. O Portable C Compiler’ de Johnson serviu de 
base para várias implementações de C em novas plataformas.

Durante muito tempo C foi distribuído juntamente com a versão 5 do UNIX. Isso, aliado ao 
fato de que um código produzido em uma máquina era facilmente recompilado em outra, causou 
uma popularização de C, tornando necessária uma padronização. Essa padronização se deu em 
1983, quando foi estabelecido um padrão pelo ANSI (American National Standard Institute).

O inicio de sua popularidade se deu por C ser altamente portável. Um código escrito em uma 
máquina pode facilmente ser compilado em outra. Hoje em dia, essa vantagem não fica tão 
evidente, pois existe uma certa padronização . Mas até poucos anos atrás havia uma grande 
diversidade de hardware, o que causava problemas de incompatibilidade com diversas linguagens. 
Os arquitetos de C pensaram nisso, criando uma linguagem que ao invés de comandos, usa funções 
para a entrada e saída de dados. Isso é bem explicado em Funções de E/S.

A linguagem C tornou-se uma das linguagens de programação mais usadas. Porém, encontra seus 
limites quando o tamanho de um projeto ultrapassa um certo ponto. Ainda que este limite possa 
variar de projeto para projeto, quanto o tamanho de um programa se encontra entre 25.000 e 100.000 
linhas, torna-se problemático o gerenciamento, tendo em vista que é difícil compreende-lo como 
um todo. Para resolver este problema, em 1980, enquanto trabalhava nos laboratórios da Bell, 
em Murray Bill, New Jersey, Bjarne Stroustrup acrescentou várias extensões à linguagem C e chamou 
inicialmente esta nova linguagem de “C com classes”. Entretanto, em 1983, o nome foi mudado para C++. 
Muitas adições foram feitas pós-Stroustrup para que a linguagem C pudesse suportar a programação 
orientada a objetos, às vezes referenciada como POO.

Hoje em dia mesmo após todo esse tempo, C continua sendo bastante utilizada. A linguagem C++ , 
implementada a partir de C é a linguagem mais usada para desenvolvimento de aplicações comerciais. 
Como C++ é basicamente a linguagem C melhorada e com orientação a objetos, o conhecimento de C é 
essencial para o domínio dessa outra linguagem.

A popularização do ambiente Windows criou um outro uso à C. A criação de DLLs, feita através dessa 
linguagem tem sustentado muito programador . Graças à portabilidade já discutida antes, C foi a 
escolha lógica para esse uso.

Frequentemente referenciada como uma linguagem de nível médio, posicionando-se entre o assembler 
(baixo nível) e o Pascal (alto nível) Uma das razões da invenção da linguagem C foi dar ao programador 
uma linguagem de alto nível que poderia ser utilizada como uma substituta para a linguagem assembly. 
Entretanto, ainda que a linguagem C possua estruturas de controle de alto nível, como é encontrado na 
Pascal, ela também permite que o programador manipule bits, bytes e endereços de uma maneira mais 
proximamente ligada à máquina, ao contrário da abstração apresentadas por outras linguagens de alto 
nível. Por esse motivo, a linguagem C tem sido ocasionalmente chamada de “código assembly de alto 
nível”. Por sua natureza dupla, a linguagem C permite que sejam criado programas rápidos e eficientes 
sem a necessidade de se recorrer a linguagem SCHILDT, H. Turbo C++: guia do usuário. São Paulo : 
Makron Books,1992.

A filosofia que existe por trás da linguagem C é que o programador sabe realmente o que está fazendo. 
Por esse motivo, a linguagem C quase nunca “colocasse no caminho” do programador, deixando-o livre 
para usar (ou abusar) dela de qualquer forma que queira. Existe uma pequena verificação de erro de 
execução runtime error. Por exemplo, se por qualquer motivo você quiser sobrescrever a memória na qual 
o seu programa está atualmente residindo, o compilador nada fará para impedi-lo. O motivo para essa 
“liberdade na programação” é permitir ao compilador C criar códigos muito rápidos e eficientes, já 
que ele deixa a responsabilidade da verificação de erros para você. Em outras palavras, a linguagem C 
considera que você é hábil o bastante para adicionar suas próprias verificações de erro quando necessário. 
Quando C++ foi inventado, Bjarne Stroustrup sabia que era importante manter o espírito original da 
linguagem C, incluindo a eficiência, a natureza de nível médio e a filosofia de que o programador, 
não a linguagem, está com as responsabilidades, enquanto, ao mesmo tempo, acrescentava o suporte à 
programação orientada a objetos. Assim, o C++ proporciona ao programador a liberdade e o controle da 
linguagem C junto com o poder dos objetos.

% ---
% Capitulo com exemplos de comandos inseridos de arquivo externo 
% ---
\include{abntex2-modelo-include-comandos}
% ---

% ----------------------------------------------------------
% PARTE
% ----------------------------------------------------------
\part{Referenciais teóricos}
% ----------------------------------------------------------

% ---
% Capitulo de revisão de literatura
% ---
\chapter{Lorem ipsum dolor sit amet}
% ---

% ---
\section{Aliquam vestibulum fringilla lorem}
% ---

\lipsum[1]

\lipsum[2-3]

% ----------------------------------------------------------
% PARTE
% ----------------------------------------------------------
\part{Resultados}
% ----------------------------------------------------------

% ---
% primeiro capitulo de Resultados
% ---
\chapter{Lectus lobortis condimentum}
% ---

% ---
\section{Vestibulum ante ipsum primis in faucibus orci luctus et ultrices
posuere cubilia Curae}
% ---

\lipsum[21-22]

% ---
% segundo capitulo de Resultados
% ---
\chapter{Nam sed tellus sit amet lectus urna ullamcorper tristique interdum
elementum}
% ---

% ---
\section{Pellentesque sit amet pede ac sem eleifend consectetuer}
% ---

\lipsum[24]

% ----------------------------------------------------------
% Finaliza a parte no bookmark do PDF
% para que se inicie o bookmark na raiz
% e adiciona espaço de parte no Sumário
% ----------------------------------------------------------
%\phantompart

% ---
% Conclusão (outro exemplo de capítulo sem numeração e presente no sumário)
% ---
\chapter*[Conclusão]{Conclusão}
\addcontentsline{toc}{chapter}{Conclusão}
% ---

\lipsum[31-33]

% ----------------------------------------------------------
% ELEMENTOS PÓS-TEXTUAIS
% ----------------------------------------------------------
\postextual
% ----------------------------------------------------------

% ----------------------------------------------------------
% Referências bibliográficas
% ----------------------------------------------------------
\bibliography{abntex2-modelo-references}

% ----------------------------------------------------------
% Glossário
% ----------------------------------------------------------
%
% Consulte o manual da classe abntex2 para orientações sobre o glossário.
%
%\glossary

% ----------------------------------------------------------
% Apêndices
% ----------------------------------------------------------

% ---
% Inicia os apêndices
% ---
\begin{apendicesenv}

% Imprime uma página indicando o início dos apêndices
\partapendices

% ----------------------------------------------------------
\chapter{Quisque libero justo}
% ----------------------------------------------------------

\lipsum[50]

% ----------------------------------------------------------
\chapter{Nullam elementum urna vel imperdiet sodales elit ipsum pharetra ligula
ac pretium ante justo a nulla curabitur tristique arcu eu metus}
% ----------------------------------------------------------
\lipsum[55-57]

\end{apendicesenv}
% ---


% ----------------------------------------------------------
% Anexos
% ----------------------------------------------------------

% ---
% Inicia os anexos
% ---
\begin{anexosenv}

% Imprime uma página indicando o início dos anexos
\partanexos

% ---
\chapter{Morbi ultrices rutrum lorem.}
% ---
\lipsum[30]

% ---
\chapter{Cras non urna sed feugiat cum sociis natoque penatibus et magnis dis
parturient montes nascetur ridiculus mus}
% ---

\lipsum[31]

% ---
\chapter{Fusce facilisis lacinia dui}
% ---

\lipsum[32]

\end{anexosenv}

%---------------------------------------------------------------------
% INDICE REMISSIVO
%---------------------------------------------------------------------
%\phantompart
\printindex
%---------------------------------------------------------------------

\end{document}
