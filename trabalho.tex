
\documentclass[
    % -- opções da classe memoir --
    12pt,               % tamanho da fonte
    openright,          % capítulos começam em páginas ímpar
    twoside,            % para impressão em verso e anverso. Oposto a oneside
    a4paper,            % tamanho do papel. 
    % -- opções da classe abntex2 --
    %chapter=TITLE,     % títulos de capítulos convertidos em letras maiúsculas
    %section=TITLE,     % títulos de seções convertidos em letras maiúsculas
    %subsection=TITLE,  % títulos de subseções convertidos em letras maiúsculas
    %subsubsection=TITLE,% títulos de subsubseções convertidos em letras maiúsculas
    % -- opções do pacote babel --
    brazil              % o último idioma é o principal do documento
    ]{abntex2}

% ---
% Pacotes básicos 
% ---
\usepackage{palatino}            % Usa a fonte Latin Modern          
\usepackage[T1]{fontenc}        % Selecao de codigos de fonte.
\usepackage[utf8]{inputenc}     % Codificacao do documento (conversão automática dos acentos)
\usepackage{lastpage}           % Usado pela Ficha catalográfica
\usepackage{indentfirst}        % Indenta o primeiro parágrafo de cada seção.
\usepackage{color}              % Controle das cores
\usepackage{graphicx}           % Inclusão de gráficos
\usepackage{microtype}          % para melhorias de justificação
% ---
        
% ---
% Pacotes de citações
% ---
\usepackage[brazilian,hyperpageref]{backref}     % Paginas com as citações na bibl
\usepackage[alf]{abntex2cite}   % Citações padrão ABNT

% --- 
% CONFIGURAÇÕES DE PACOTES
% --- 

% ---
% Configurações do pacote backref
% Usado sem a opção hyperpageref de backref
\renewcommand{\backrefpagesname}{Citado na(s) página(s):~}
% Texto padrão antes do número das páginas
\renewcommand{\backref}{}
% Define os textos da citação
\renewcommand*{\backrefalt}[4]{
    \ifcase #1 %
        Nenhuma citação no texto.%
    \or
        Citado na página #2.%
    \else
        Citado #1 vezes nas páginas #2.%
    \fi}%
% ---

% ---
% Informações de dados para CAPA e FOLHA DE ROSTO
% ---
\titulo{Trabalho de Paradigmas de \\ Linguagens de Programação}
\autor{
    Eric Bueno Gauch
    \and
    Guilherme Gatto Gonçalves da Silva
    \and
    Bruno Estima Correia Milanesi Castanheira de Souza}
\local{Brasil}
\data{2017, v-0.0.1}
\instituicao{%
  Universidade Presbiteriana Mackenzie -- UPM
  \par
  Faculdade de Computação e Informática
  \par
  Programa de Graduação em Ciência da Computação}
\tipotrabalho{Trabalho Acadêmico}
% O preambulo deve conter o tipo do trabalho, o objetivo, 
% o nome da instituição e a área de concentração 
\preambulo{Trabalho da disciplina de Paradigma de Linguagem de Programação sobre a 
evolução das principais linguagens de programação}
% ---


% ---
% Configurações de aparência do PDF final

% alterando o aspecto da cor azul
\definecolor{blue}{RGB}{41,5,195}

% informações do PDF
\makeatletter
\hypersetup{
        %pagebackref=true,
        pdftitle={\@title}, 
        pdfauthor={\@author},
        pdfsubject={\imprimirpreambulo},
        pdfcreator={LaTeX with abnTeX2},
        pdfkeywords={abnt}{latex}{abntex}{abntex2}{trabalho acadêmico}, 
        colorlinks=true,            % false: boxed links; true: colored links
        linkcolor=black,             % color of internal links
        citecolor=blue,             % color of links to bibliography
        filecolor=magenta,              % color of file links
        urlcolor=blue,
        bookmarksdepth=4
}
\makeatother
% --- 

% --- 
% Espaçamentos entre linhas e parágrafos 
% --- 

% O tamanho do parágrafo é dado por:
\setlength{\parindent}{1.3cm}

% Controle do espaçamento entre um parágrafo e outro:
\setlength{\parskip}{0.2cm}  % tente também \onelineskip

% ---
% compila o indice
% ---
\makeindex
% ---

% ----
% Início do documento
% ----
\begin{document}

% Retira espaço extra obsoleto entre as frases.
\frenchspacing 

% ----------------------------------------------------------
% ELEMENTOS PRÉ-TEXTUAIS
% ----------------------------------------------------------
% \pretextual

% ---
% Capa
% ---
\imprimircapa
% ---

% ---
% Folha de rosto
% (o * indica que haverá a ficha bibliográfica)
% ---
\imprimirfolhaderosto*
% ---

% ---
% inserir o sumario
% ---
\clearpage
\pdfbookmark[0]{\contentsname}{toc}
\tableofcontents*
\cleardoublepage
% ---



% ----------------------------------------------------------
% ELEMENTOS TEXTUAIS
% ----------------------------------------------------------
\textual

% ----------------------------------------------------------
% Introdução (exemplo de capítulo sem numeração, mas presente no Sumário)
% ----------------------------------------------------------
\chapter*[Introdução]{Introdução}
\addcontentsline{toc}{chapter}{Introdução}
% ----------------------------------------------------------

Este primeiro trabalho da disciplina de Paradigmas de Linguagens de Programação
tem como objetivo proporcionar uma visão geral da história das linguagens de
programação utilizando as 3 linguagens de programação mais populares hoje 
segundo a classificação da TIOBE Software. É também abordado, além da história dessas
linguagens, como elas surgiram, seus criadores, fatos que os inspiraram, principais
objetivos das linguagens e suas respectivas trajetórias até o topo do ranking de
popularidade.
Nos capítulos seguintes é discutido porque algumas linguagens continuam
evoluindo e se adaptando as necessidades que surgem e outras nunca evoluem.
Por último, uma tabela com as 10 linguagens mais populares segundo a
classificação da TIOBE é apresentada.

% ----------------------------------------------------------
% PARTE
% ----------------------------------------------------------
\chapter{Evolução das Principais Linguagens de Programação}
% ----------------------------------------------------------

% ----------------------------------------------------------
% CAPÍTULO
% ----------------------------------------------------------
\section{JAVA}

Em 1990, a empresa Sun Microsystems tinha um projeto, utilizando a linguagem
C++, de ligar várias interfaces e viabilizar a intercomunicação entre diversos
dispositivos.  A equipe que era liderada por James Gosling ficou responsável
por realizar esse feito. Diante das limitações tecnológicas da época, o grupo
criou a linguagem GreenTalk que tinha como objetivo a intercomunicação de
múltiplos aparelhos.  GreenTalk se tornou um dos maiores projetos da Sun
Microsystems e rapidamente foi rebatizado, em 1991, com o nome de Oak. Uma das
teorias é de que o nome foi escolhido devido às várias árvores de carvalho que
compunham a vista da sala de Gosling. Enquanto Tim Berners-Lee criava o HTML,
uma linguagem interativa (característica também presente na linguagem Oak) para
WEB, a ideia de união de esforços surgiu dando inicio ao projeto WebRunner, um
navegador cuja proposta era implementar todas as funcionalidades do Star7, um
dispositivo capaz de controlar periféricos domésticos, mas dessa fez para WEB.

A mudança de nome da linguagem ocorreu por causa de um problema durante o
processo de registro.  Uma outra tecnologia com o nome Oak já estava
registrado, obrigando o time de Gosling a se reunir para definir um novo nome
para a linguagem. O nome Java surgiu por causa do amor da equipe pelo café
forte cultivado na Ilha de Java. 

A linguagem começou a popularizar em 2004 quando a NASA criou um pequeno robô
para mapear o solo de Marte. O código do robô foi escrito em Java para facilitar
a comunicação com a Terra. Em 2006 com a popularização da comunidade
\textit{open source} a linguagem Java entrou para a comunidade, se tornando uma
linguagem \textit{open source}.  Em 2009 a empresa Sun Microsystems foi vendida
para Oracle por 7.4 bilhões de dólares, tornando a linguagem Java propriedade da
Oracle. Hoje em dia a linguagem java está por toda parte, em celulares,
aplicativos de banco, televisores, relógios, leitores de cartões, entre outros.
Java é hoje a linguagem mais popular e usada no mundo.  Porque java?  Java roda
em uma maquina virtual, o que faz com que qualquer sistema operacional suporte
aplicações criadas nessa linguagem. Conhecida pela frase “Escreva uma vez,
execute em qualquer lugar”, foi o que tornou a linguagem tão popular.

% ----------------------------------------------------------
% CAPÍTULO
% ----------------------------------------------------------
\section{C}

O desenvolvimento inicial de C ocorreu no ATT Bell Labs entre 1969 e 1973. A
linguagem foi chamada “C”, porque suas características foram obtidas a partir
de uma linguagem anteriormente chamado de ” B”, que era versão reduzida da
linguagem de programação BCPL.6.

A versão original PDP-11 do sistema Unix foi desenvolvido em Assembly. Em 1973,
com a adição dos tipos \textit{struct}, a linguagem C tornou-se poderosa o
suficiente para que a maior parte do \textit{kernel} do Unix fosse reescrito em
C. Este foi um dos primeiros núcleos de sistemas operacionais implementadas em
uma linguagem diferente da linguagem Assembly. 

Durante muito tempo C foi distribuído juntamente com a versão 5 do UNIX. Isso,
aliado ao fato de que um código produzido em uma máquina era facilmente
recompilado em outra, causou uma popularização de C, tornando necessária uma
padronização. Essa padronização se deu em 1983, quando foi estabelecido um
padrão pelo ANSI (American National Standard Institute).

Hoje em dia mesmo após todo esse tempo, C continua sendo bastante utilizada. A
linguagem C++, implementada a partir de C é a linguagem mais usada para
desenvolvimento de aplicações comerciais.  Como C++ é basicamente a linguagem C
melhorada e com orientação a objetos, o conhecimento de C é essencial para o
domínio dessa outra linguagem. A popularização do ambiente Windows criou um
outro uso à C. A criação de DLLs, feita através dessa linguagem tem sustentado
muito programador . Graças à portabilidade já discutida antes, C foi a escolha
lógica para esse uso.

A filosofia que existe por trás da linguagem C é que o programador sabe
realmente o que está fazendo.  Por esse motivo, a linguagem C quase nunca
“colocasse no caminho” do programador, deixando-o livre para usar (ou abusar)
dela de qualquer forma que queira. O motivo para essa “liberdade na
programação” é permitir ao compilador C criar códigos muito rápidos e
eficientes, já que ele deixa a responsabilidade da verificação de erros para
você. Em outras palavras, a linguagem C considera que você é hábil o bastante
para adicionar suas próprias verificações de erro quando necessário. 

% ----------------------------------------------------------
% CAPÍTULO
% ----------------------------------------------------------
\clearpage
\section{C++}

O C++ foi inicialmente desenvolvido por Bjarne Stroustrup dos Bell Labs,
durante a década de 1980 com o objetivo implementar uma versão distribuída do
núcleo Unix. Como o Unix era escrito em C, dever-se-ia manter a
compatibilidade, ainda que adicionando novos recursos. Alguns dos desafios
incluíam simular a infraestrutura da comunicação entre processos num sistema
distribuído ou de memória compartilhada e escrever \textit{drivers} para tal
sistema. O C foi escolhido como base de desenvolvimento da nova linguagem pois
possuía uma proposta de uso genérico, era rápido e também portável para
diversas plataformas. Algumas outras linguagens que também serviram de
inspiração para o cientista da computação foram ALGOL 68, Ada, CLU e ML.

Ainda em 1983 o nome da linguagem foi alterado de C with Classes para C++.
Antes implementada usando um pré-processador, a linguagem passou a exigir um
compilador próprio, escrito pelo próprio Stroustrup. Assim como a linguagem,
sua biblioteca padrão também sofreu melhorias ao longo do tempo. Sua primeira
adição foi a biblioteca de E/S, e posteriormente a Standard Template Library
(STL); ambas tornaram-se algumas das principais funcionalidades que
distanciaram a linguagem em relação a C.  Depois de anos de trabalho, um comitê
unificado ANSI/ISO padronizou o C++ em 1998 (ISO/IEC 14882:1998). Após alguns
anos foram reportados defeitos e imprecisões no documento, e uma correção foi
lançada em 2003.

Pode-se dizer que C++ foi a única linguagem entre tantas outras que obteve
sucesso como uma sucessora à linguagem C, inclusive servindo de inspiração para
outras linguagens como Java, a IDL de CORBA e C\#.

% ----------------------------------------------------------
% PARTE
% ----------------------------------------------------------
\chapter{Linguagens “Mãe”}
% ----------------------------------------------------------

Antes de definirmos uma linguagem mãe, vamos ao antigo e velho dicionário para
entendermos separadamente o que significa cada uma dessas palavras: 

\begin{description} \item[$\cdot$ Mãe] Mulher que deu à luz, que cria ou criou
um ou mais filhos.  \item[$\cdot$ Linguagem] Qualquer meio sistemático de
comunicar ideias ou sentimentos através de signos convencionais, sonoros,
gráficos, gestuais etc.
\end{description}

Tendo a definição dessas duas palavras e levando ao contexto tecnológico,
podemos concluir que linguagem mãe é aquela que dá origem a novas linguagens,
que de acordo com nosso contexto são as linguagens de programação.  Para que
essa definição fique mais clara, vamos dar como exemplo a linguagem de
programação C. Devido a sua enorme popularidade e uso no mercado, a linguagem C
serviu como origem para novas linguagens a fim de evoluir o modo como é
realizada a programação. Uma dessas linguagens é o C++, que foi baseado na
linguagem C mas com a diferença de ser uma linguagem orientada a objeto.
Concluímos então, que toda e qualquer linguagem que dê origem as novas
linguagens são consideradas linguagens mãe.

% ----------------------------------------------------------
% PARTE
% ----------------------------------------------------------
\chapter{Por Que Algumas Linguagens Sempre Evoluem Enquanto Outras Nunca Evoluem?}
% ----------------------------------------------------------

De acordo com pesquisas e profissionais experientes da área de tecnologia,
podemos notar um ponto em comum, ambos concordam que linguagens mais difundidas
são apenas variações sintáticas de linguagens mais antigas que,
consequentemente, não contribuem com nada de novo.  Para entendermos melhor esse
assunto, vamos referenciar uma apresentação de Todd Proebsting, pesquisador da
Microsoft, em sua apresentação denominada Disruptive Programming Language
Technologies.

Proebsting argumenta que a criação de novas linguagens de programação é como
melhorar a produtividade do programador, já que o hardware hoje em dia é
suficientemente poderoso e não requer que compiladores otimizem massivamente o
código e dados de um programa.

O ponto da apresentação não é mostrar como criar novas linguagens de
programação, mas como elas podem ter sucesso onde outras falharam.  Proebsting
ainda enfatiza que algoritmos e modelagens apropriadas são suficientes para o
desenvolvimento de programas eficientes e que criadores de linguagens deveriam
se preocupar em como ajudar programadores a realizar bons programas de forma
eficiente.

Com isso podemos concluir que a evolução das linguagens de programação se dão
devido a forma na qual elas resolvem problemas atuais e ao mesmo tempo deixam o
ato de programar mais eficiente. Tendo isso como afirmação, podemos entender o
porque de linguagens antigas estarem sendo menos utilizadas.

% ----------------------------------------------------------
% PARTE
% ----------------------------------------------------------
\chapter{As Linguagens Mais Utilizadas Hoje}
% ----------------------------------------------------------

% ----------------------------------------------------------
% CAPÍTULO
% ----------------------------------------------------------
\section{Classificação}

\begin{center}
\begin{tabular}{ | l | l | l | p{5cm} |}
\hline
Posição & Linguagem de Programação \\ \hline
1 & Java \\ \hline
2 & C \\ \hline
3 & C++ \\ \hline
4 & C\# \\ \hline
5 & Python \\ \hline
6 & Visual Basic .NET \\ \hline
7 & PHP \\ \hline
8 & JavaScript \\ \hline
9 & Perl \\ \hline
10 & Ruby \\ \hline
\end{tabular}
\end{center}

% ----------------------------------------------------------
% CAPÍTULO
% ----------------------------------------------------------
\section{Como a Classificação Funciona?}

Para ser considerada na classificação das linguagens de programação da TIOBE
Software, a linguagem deve cumprir os três requisitos abaixo:

\begin{description}
\item[$\cdot$]A linguagem deve ter uma entrada própria no Wikipedia e o
Wikipedia deve claramente considerá-la uma linguagem de programação.
\item[$\cdot$]A linguagem de programação deve ser Turing-completa.
\item[$\cdot$]A linguagem deve ter no mínimo 5000 buscas por \textit{<language>
programming}

A posição de uma linguagem na classificação é calculada pelo número de buscas
nos mais populares mecanismos de pesquisa.

\end{description}
% ----------------------------------------------------------
% Referências bibliográficas
% ----------------------------------------------------------
\bibliography{abntex2-modelo-references}

\end{document}
