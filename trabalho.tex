%% abtex2-modelo-trabalho-academico.tex, v-1.9.2 laurocesar
%% Copyright 2012-2014 by abnTeX2 group at http://abntex2.googlecode.com/ 
%%
%% This work may be distributed and/or modified under the
%% conditions of the LaTeX Project Public License, either version 1.3
%% of this license or (at your option) any later version.
%% The latest version of this license is in
%%   http://www.latex-project.org/lppl.txt
%% and version 1.3 or later is part of all distributions of LaTeX
%% version 2005/12/01 or later.
%%
%% This work has the LPPL maintenance status `maintained'.
%% 
%% The Current Maintainer of this work is the abnTeX2 team, led
%% by Lauro César Araujo. Further information are available on 
%% http://abntex2.googlecode.com/
%%
%% This work consists of the files abntex2-modelo-trabalho-academico.tex,
%% abntex2-modelo-include-comandos and abntex2-modelo-references.bib
%%

% ------------------------------------------------------------------------
% ------------------------------------------------------------------------
% abnTeX2: Modelo de Trabalho Academico (tese de doutorado, dissertacao de
% mestrado e trabalhos monograficos em geral) em conformidade com 
% ABNT NBR 14724:2011: Informacao e documentacao - Trabalhos academicos -
% Apresentacao
% ------------------------------------------------------------------------
% ------------------------------------------------------------------------

\documentclass[
    % -- opções da classe memoir --
    12pt,               % tamanho da fonte
    openright,          % capítulos começam em pág ímpar (insere página vazia caso preciso)
    twoside,            % para impressão em verso e anverso. Oposto a oneside
    a4paper,            % tamanho do papel. 
    % -- opções da classe abntex2 --
    %chapter=TITLE,     % títulos de capítulos convertidos em letras maiúsculas
    %section=TITLE,     % títulos de seções convertidos em letras maiúsculas
    %subsection=TITLE,  % títulos de subseções convertidos em letras maiúsculas
    %subsubsection=TITLE,% títulos de subsubseções convertidos em letras maiúsculas
    % -- opções do pacote babel --
    brazil              % o último idioma é o principal do documento
    ]{abntex2}

% ---
% Pacotes básicos 
% ---
\usepackage{palatino}            % Usa a fonte Latin Modern          
\usepackage[T1]{fontenc}        % Selecao de codigos de fonte.
\usepackage[utf8]{inputenc}     % Codificacao do documento (conversão automática dos acentos)
\usepackage{lastpage}           % Usado pela Ficha catalográfica
\usepackage{indentfirst}        % Indenta o primeiro parágrafo de cada seção.
\usepackage{color}              % Controle das cores
\usepackage{graphicx}           % Inclusão de gráficos
\usepackage{microtype}          % para melhorias de justificação
% ---
        
% ---
% Pacotes adicionais, usados apenas no âmbito do Modelo Canônico do abnteX2
% ---
\usepackage{lipsum}             % para geração de dummy text
% ---

% ---
% Pacotes de citações
% ---
\usepackage[brazilian,hyperpageref]{backref}     % Paginas com as citações na bibl
\usepackage[alf]{abntex2cite}   % Citações padrão ABNT

% --- 
% CONFIGURAÇÕES DE PACOTES
% --- 

% ---
% Configurações do pacote backref
% Usado sem a opção hyperpageref de backref
\renewcommand{\backrefpagesname}{Citado na(s) página(s):~}
% Texto padrão antes do número das páginas
\renewcommand{\backref}{}
% Define os textos da citação
\renewcommand*{\backrefalt}[4]{
    \ifcase #1 %
        Nenhuma citação no texto.%
    \or
        Citado na página #2.%
    \else
        Citado #1 vezes nas páginas #2.%
    \fi}%
% ---

% ---
% Informações de dados para CAPA e FOLHA DE ROSTO
% ---
\titulo{Trabalho de Paradigmas de \\ Linguagens de Programação}
\autor{
    Eric Bueno Gauch
    \and
    Guilherme Gatto Gonçalves da Silva
    \and
    Bruno Estima Correia Milanesi Castanheira de Souza}
\local{Brasil}
\data{2017, v-0.0.1}
\instituicao{%
  Universidade Presbiteriana Mackenzie -- UPM
  \par
  Faculdade de Computação e Informática
  \par
  Programa de Graduação em Ciência da Computação}
\tipotrabalho{Trabalho Acadêmico}
% O preambulo deve conter o tipo do trabalho, o objetivo, 
% o nome da instituição e a área de concentração 
\preambulo{Trabalho da disciplina de Paradigma de Linguagem de Programação sobre a 
evolução das principais linguagens de programação}
% ---


% ---
% Configurações de aparência do PDF final

% alterando o aspecto da cor azul
\definecolor{blue}{RGB}{41,5,195}

% informações do PDF
\makeatletter
\hypersetup{
        %pagebackref=true,
        pdftitle={\@title}, 
        pdfauthor={\@author},
        pdfsubject={\imprimirpreambulo},
        pdfcreator={LaTeX with abnTeX2},
        pdfkeywords={abnt}{latex}{abntex}{abntex2}{trabalho acadêmico}, 
        colorlinks=true,            % false: boxed links; true: colored links
        linkcolor=black,             % color of internal links
        citecolor=blue,             % color of links to bibliography
        filecolor=magenta,              % color of file links
        urlcolor=blue,
        bookmarksdepth=4
}
\makeatother
% --- 

% --- 
% Espaçamentos entre linhas e parágrafos 
% --- 

% O tamanho do parágrafo é dado por:
\setlength{\parindent}{1.3cm}

% Controle do espaçamento entre um parágrafo e outro:
\setlength{\parskip}{0.2cm}  % tente também \onelineskip

% ---
% compila o indice
% ---
\makeindex
% ---

% ----
% Início do documento
% ----
\begin{document}

% Retira espaço extra obsoleto entre as frases.
\frenchspacing 

% ----------------------------------------------------------
% ELEMENTOS PRÉ-TEXTUAIS
% ----------------------------------------------------------
% \pretextual

% ---
% Capa
% ---
\imprimircapa
% ---

% ---
% Folha de rosto
% (o * indica que haverá a ficha bibliográfica)
% ---
\imprimirfolhaderosto*
% ---

% ---
% inserir o sumario
% ---
\clearpage
\pdfbookmark[0]{\contentsname}{toc}
\tableofcontents*
\cleardoublepage
% ---



% ----------------------------------------------------------
% ELEMENTOS TEXTUAIS
% ----------------------------------------------------------
\textual

% ----------------------------------------------------------
% Introdução (exemplo de capítulo sem numeração, mas presente no Sumário)
% ----------------------------------------------------------
\chapter*[Introdução]{Introdução}
\addcontentsline{toc}{chapter}{Introdução}
% ----------------------------------------------------------

Este primeiro trabalho da disciplina de Paradigmas de Linguagens de Programação
tem como objetivo introduzir as principais linguagens de programação em uso 
atualmente bem como  

% ----------------------------------------------------------
% PARTE
% ----------------------------------------------------------
\part{Evolução das Principais Linguagens de Programação}
% ----------------------------------------------------------

\chapter{JAVA}

Em 1990, a empresa Sun Microsystems tinha um projeto, utilizando a linguagem
C++, de ligar várias interfaces e viabilizar a intercomunicação entre diversos
dispositivos.  A equipe que era liderada por James Gosling ficou responsável
por realizar esse feito. Diante das limitações tecnológicas da época, o grupo
criou a linguagem GreenTalk que tinha como objetivo a intercomunicação de
múltiplos aparelhos.  GreenTalk se tornou um dos maiores projetos da Sun
Microsystems e rapidamente foi rebatizado, em 1991, com o nome de Oak. Uma das
teorias é de que o nome foi escolhido devido às várias árvores de carvalho que
compunham a vista da sala de Gosling. Enquanto Tim Berners-Lee criava o HTML,
uma linguagem interativa (característica também presente na linguagem Oak) para
WEB, a ideia de união de esforços surgiu dando inicio ao projeto WebRunner, um
navegador cuja proposta era implementar todas as funcionalidades do Star7, um
dispositivo capaz de controlar periféricos domésticos, mas dessa fez para WEB.

A mudança de nome da linguagem ocorreu por causa de um problema durante o
processo de registro.  Uma outra tecnologia com o nome Oak já estava
registrado, obrigando o time de Gosling a se reunir para definir um novo nome
para a linguagem. O nome Java surgiu por causa do amor da equipe pelo café
forte cultivado na Ilha de Java. 

A linguagem começou a popularizar em 2004 quando a NASA criou um pequeno robô
para mapear o solo de Marte. O código do robô foi escrito em Java para
facilitar a comunicação com a Terra. Em 2006 com a popularização da comunidade
\textit{open source} a linguagem Java entrou para a comunidade, se tornando uma
linguagem open source.  Em 2009 a empresa Sun Microsystems foi vendida para
Oracle por 7.4 bilhões de dólares, tornando a linguagem Java propriedade da
Oracle. Hoje em dia a linguagem java está por toda parte, em celulares,
aplicativos de banco, televisores, relógios, leitores de cartões, entre outros.
Java é hoje a linguagem mais popular e usada no mundo.  Porque java?  Java roda
em uma maquina virtual, o que faz com que qualquer sistema operacional suporte
aplicações criadas nessa linguagem. Conhecida pela frase “Escreva uma vez,
execute em qualquer lugar”, foi o que tornou a linguagem tão popular.

\chapter{C}

O desenvolvimento inicial de C ocorreu no ATT Bell Labs entre 1969 e 1973. A
linguagem foi chamada “C”, porque suas características foram obtidas a partir
de uma linguagem anteriormente chamado de ” B”, que era versão reduzida da
linguagem de programação BCPL.6.

A versão original PDP-11 do sistema Unix foi desenvolvido em Assembly. Em 1973,
com a adição dos tipos \textit{struct}, a linguagem C tornou-se poderosa o
suficiente para que a maior parte do \textit{kernel} do Unix fosse reescrito em
C. Este foi um dos primeiros núcleos de sistemas operacionais implementadas em
uma linguagem diferente da linguagem Assembly. 

Durante muito tempo C foi distribuído juntamente com a versão 5 do UNIX. Isso,
aliado ao fato de que um código produzido em uma máquina era facilmente
recompilado em outra, causou uma popularização de C, tornando necessária uma
padronização. Essa padronização se deu em 1983, quando foi estabelecido um
padrão pelo ANSI (American National Standard Institute).

Hoje em dia mesmo após todo esse tempo, C continua sendo bastante utilizada. A
linguagem C++, implementada a partir de C é a linguagem mais usada para
desenvolvimento de aplicações comerciais.  Como C++ é basicamente a linguagem C
melhorada e com orientação a objetos, o conhecimento de C é essencial para o
domínio dessa outra linguagem. A popularização do ambiente Windows criou um
outro uso à C. A criação de DLLs, feita através dessa linguagem tem sustentado
muito programador . Graças à portabilidade já discutida antes, C foi a escolha
lógica para esse uso.

A filosofia que existe por trás da linguagem C é que o programador sabe
realmente o que está fazendo.  Por esse motivo, a linguagem C quase nunca
“colocasse no caminho” do programador, deixando-o livre para usar (ou abusar)
dela de qualquer forma que queira. O motivo para essa “liberdade na
programação” é permitir ao compilador C criar códigos muito rápidos e
eficientes, já que ele deixa a responsabilidade da verificação de erros para
você. Em outras palavras, a linguagem C considera que você é hábil o bastante
para adicionar suas próprias verificações de erro quando necessário. 

\chapter{C++}

% ----------------------------------------------------------
% PARTE
% ----------------------------------------------------------
\part{Linguagens “Mãe”}
% ----------------------------------------------------------

% ----------------------------------------------------------
% PARTE
% ----------------------------------------------------------
\part{Por Que Algumas Linguagens Deixam de Evoluir?}
% ----------------------------------------------------------

% ----------------------------------------------------------
% PARTE
% ----------------------------------------------------------
\part{Por Que Algumas Linguagens Sempre Evoluem?}
% ----------------------------------------------------------

% ----------------------------------------------------------
% PARTE
% ----------------------------------------------------------
\part{As Linguagens Mais Utilizadas Hoje}
% ----------------------------------------------------------

% ----------------------------------------------------------
% Referências bibliográficas
% ----------------------------------------------------------
\bibliography{abntex2-modelo-references}

\end{document}
